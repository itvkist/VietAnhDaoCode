% !TeX root = ../main.tex
\documentclass[./../main.tex]{subfiles}

\begin{document}
Hệ thống sử dụng công nghệ \acrshort{cms} Directus để xây dựng, phát triển cho backend, sử dụng công nghệ ReactJS để phát triển frontend, tránh việc tải lại trang web nhằm tăng trải nghiệm người dùng. Cơ sở dữ liệu được sử dụng là PostgreSQL. Ngoài ra, để quản lý mã lệnh để chạy chương trình, hệ thống sử dụng công cụ Github. Chương này sẽ tập trung giới thiệu sơ lược về các công nghệ trên và các đặc điểm nổi bật của chúng.

\section{Javascript}
JavaScript (JS) là ngôn ngữ lập trình có kích thước nhẹ, được thông dịch hoặc thực thi tức thời và được biết đến nhiều nhất là ngôn ngữ kịch bản cho việc xây dựng các trang web. JavaScript là ngôn ngữ lập trình được nhà phát triển sử dụng để tạo trang web có khả năng tương tác.\\
JavaScript có những ưu điểm nổi trội sau:
\begin{itemize}
    \item Dễ học và là mã nguồn bậc cao dễ đọc hiểu.
    \item Có thể thực thi tức thời và có khả năng phát hiện, khắc phục lỗi.
    \item Hoạt động được trên nhiều trình duyệt khác nhau.
    \item Hỗ trợ nguời dùng tương tác với giao diện.
\end{itemize}
Tuy nhiên, nó cũng có một số nhược điểm:
\begin{itemize}
    \item Dễ bị khai thác.
    \item Có khả năng thực thi mã độc vào hệ thống ở phía người dùng.
\end{itemize}
Nhìn chung, việc sử dụng JavaScript để phát triển frontend cho hệ thống giúp cho công việc lập trình, tìm và sửa lỗi, kiểm thử được diễn ra khá thuận tiện và dễ dàng. Nhưng để đảm bảo tính bảo mật cho hệ thống thì cần có thêm các biện pháp phòng thủ.

\section{ReactJS}
React \cite{react} là một thư viện JavaScript được dùng để xây dựng giao diện người dùng (\acrshort{ui}) bằng ngôn ngữ JavaScript (hoặc Typescript) do Facebook tạo ra vào tháng 5 năm 2013. React là một trong những framework phổ biến nhất hiện nay được sử dụng để xây dựng và phát triển ứng dụng web. React cải thiện công việc cho lập trình viên bằng cách cung cấp nhiều hàm phụ trợ cũng như các công cụ phục vụ cho tìm và gỡ lỗi. React có thể được sử dụng để phát triển cả ứng dụng web lẫn mobile. Trong hệ thống này, ReactJS cùng với các thư viện hỗ trợ khác được sử dụng để phát triển frontend. Khi gặp vấn đề cho việc sử dụng công cụ vào dự án, các hỗ trợ từ cộng đồng to lớn của framework giúp giải quyết, góp phần cải thiện hiệu năng và chi phí chung cho công đoạn phát triển.

Các tính năng nổi bật:
\begin{itemize}
    \item Các ứng dụng React thường được dựng lên từ các component (thành phần). Các thành phần có chứa logic và luồng điều khiển riêng có khả năng liên kết với nhau và với luồng hoạt động chung. Do đó lập trình viên có thể tái sử dụng, từ đó làm giảm thời gian phát triển, tăng tính thẩm mỹ cho giao diện vì có tính đồng nhất.
    \item React sử dụng Virtual \acrshort{dom} (\acrlong{dom}), hay còn gọi là cây đối tượng ảo, để tăng hiệu suất trong trang web. Có nghĩa là bằng việc so sánh sự thay đổi trạng thái, hình thái các component, Virtual \acrshort{dom} sẽ chỉ cập nhật các mục trong cây \acrshort{dom} của trang web, thay vì cập nhật lại toàn bộ trang.
    \item React có cấu trúc dữ liệu đơn luồng, có thể hiểu là các trang web sẽ chứa các component độc lập, các component cha (tập các component được gộp trong một component cha). Vì vậy, dữ liệu được truyền đi hay trả về theo một hướng, giúp cho việc tìm và sửa lỗi dễ dàng.
    \item React cho phép ứng dụng \acrshort{jsx} (Javascript + \acrshort{xml}) biến đổi cú pháp dạng như \acrshort{xml} thành JavaScript. Giúp người lập trình có thể lập trình ứng dụng React bằng cú pháp của \acrshort{xml} thay vì sử dụng JavaScript.
\end{itemize}

Ngoài ra, để tập hợp và lưu trữ dữ liệu ở phía frontend, phần mềm sử dụng tính năng được tích hợp sẵn của ReactJs là React Context, một trong những cách phổ biến để quản lý trạng thái trên toàn bộ hệ thống. Với ý tưởng là chia giao diện thành các phần tử nhỏ hơn để dễ dàng tái sử dụng và bảo trì thì khi ứng dụng phát triển lớn hơn, cũng đồng nghĩa với việc có nhiều tầng lớp phần tử cha, phần tử con lồng nhau, nếu cứ truyền từng dữ liệu từ cha xuống con thì việc quản lý chúng sẽ trở nên vô cùng phức tạp, thậm chí có thể gây ra lỗi nếu xử lý không đúng. Để giải quyết vấn đề này, Context sẽ lưu những trạng thái chung được nhiều phần tử sử dụng, những hàm có tác động tới các trạng thái đó và khi cần sử dụng thì chỉ cần tương tác trực tiếp với nơi lưu trữ của Context thay vì phải tương tác qua lại giữa nhiều tầng phần tử trên dưới, điều này giúp việc quản lý trạng thái dữ liệu đơn giản và rõ ràng hơn. Trong ứng dụng có nhiều dữ liệu như sắn, bệnh về sắn, những đề xuất trên sàn thương mại sắn, dữ liệu người dùng,.. thì việc phải quản lý chúng và chia sẻ thông tin với các thành phần con là việc tốn nhiều chi phí, nhưng bằng cách tập trung dữ liệu vào Context, lập trình viên có thể dễ dàng điều khiển luồng dữ liệu truyền đến đúng nơi, việc quản lý tập trung dữ liệu cũng giúp cho việc tìm và gỡ lỗi thông tin thuận tiện hơn.

\section{Directus}
Directus \cite{directus} đã bắt đầu như là Headless \acrshort{cms} một thập kỷ trước khi thuật ngữ “Headless \acrshort{cms}” được ra đời vào năm 2014. Directus cá nhân hóa dữ liệu giữa các cơ quan, tổ chức với một nền tảng trực quan, dễ sử dụng cho cả người dùng kỹ thuật và phi kỹ thuật. Directus có thể cung cấp các dạng dịch vụ đám mây, nó loại bỏ yêu cầu về việc thiết kế kiến trúc, cung cấp và bảo trì cơ sở hạ tầng của riêng mỗi dự án. Sử dụng ứng dụng không cần lập trình, người dùng có thể bắt đầu tạo trải nghiệm kỹ thuật số có khả năng tùy biến cao.

\subsection{Headless CMS}
Trước tiên, để hiểu cách hoạt động và sử dụng Directus cần phải biết Headless \acrshort{cms} là gì \cite{headlesscms}. \acrshort{cms} (\acrlong{cms}) là hệ thống quản lý nội dung, có 2 loại là Traditional \acrshort{cms} (\acrshort{cms} truyền thống) và Headless \acrshort{cms} (\acrshort{cms} không giao diện). Trong khi \acrshort{cms} truyền thống như WordPress, Wix hoặc Drupal có mọi thứ được đóng gói với nhau, và về mặt kỹ thuật, frontend (design, layout), backend (source code) và tầng lưu trữ (database) đều liên kết với nhau thành một khối chặt chẽ, thì Headless \acrshort{cms} chỉ tập trung về phía backend và lưu trữ dữ liệu, tương tác với frontend qua \acrshort{api}. Nó tách rời hệ thống quản lý nội dung phía sau và lớp trình bày phía trước. Có nghĩa là, mọi giao diện hiển thị trang web sẽ bị loại bỏ và người lập trình chỉ cần quản lý dữ liệu và nội dung không liên quan đến frontend.

Bằng cách tách biệt việc tạo, quản lý, lưu trữ nội dung khỏi lớp trình bày, nội dung có thể được truy cập liền mạch thông qua ứng dụng mà không cần lập trình, đồng thời có thể xuất bản lên bất kỳ thiết bị nào được kết nối qua IoT. Thông qua bảng điều khiển được tích hợp sẵn, người dùng có thể tùy ý điều chỉnh các cấu trúc của các bảng dữ liệu, quyền truy cập của từng vai trò và thêm, sửa, xóa các dữ liệu đó. Để giao tiếp với frontend, Headless \acrshort{cms} cũng hỗ trợ cung cấp các \acrshort{api} để phía máy khách có thể lấy dữ liệu, tạo, sửa, xóa các bản ghi trong cơ sở dữ liệu.

Các đặc điểm của Headless \acrshort{cms}:
\begin{itemize}
    \item Tiết kiệm thời gian và chi phí: Nội dung có thể được tái sử dụng, hạn chế phụ thuộc vào chi phí cho cơ sở hạ tầng và các chuyên gia thiết kế.
    \item Tối đa hóa tính linh hoạt: Trao đổi với frontend qua \acrshort{api} nên giao diện có thể được thiết kế tùy theo nhu cầu người dùng.
    \item Đơn giản hóa quy trình công việc: Chỉ tập trung cho việc quản lý dữ liệu vậy nên không bị ảnh hưởng của việc thiết kế giao diện.
    \item Cải thiện bảo mật: Do trao đổi thông qua \acrshort{api}, việc hệ thống backend bị tấn công trực diện hay ảnh hưởng bởi tấn công từ phía frontend cũng giảm thiểu, ít nghiêm trọng hơn so với \acrshort{cms} truyền thống.
    \item Kích hoạt khả năng đa ngôn ngữ: Giúp cho việc lưu trữ và hiển thị với nhiều ngôn ngữ, tăng tính linh hoạt cho dự án.
\end{itemize}

\subsection{Directus và Headless \acrshort{cms}}
Directus đã áp dụng Headless \acrshort{cms} trước khi thuật ngữ này ra đời. Nó giúp quản lý nội dung, người dùng liền mạch, linh hoạt và không có giới hạn hoặc rào cản trong việc thiết kế. Là ứng dụng có mã nguồn mở, với GPLv3 License và có thể tự lưu trữ và quản lý dữ liệu cục bộ (self-hosted) hoặc chạy trên hệ thống đám mây tiêu chuẩn, Directus cũng cung cấp phiên bản thương mại tùy theo nhu cầu người dùng. Nó cũng cho phép sử dụng nhiều cơ sở dữ liệu phổ biến (SQLite, MySQL, PostgreSQL,...) và chạy trên Docker. 

Các tính năng nổi bật:
\begin{itemize}
    \item Bộ công cụ với nhiều tính năng được cập nhật thường xuyên trong ứng dụng Directus để nâng cao trải nghiệm khi sử dụng.
    \item Phần giao diện và bố cục hoàn toàn có thể tùy chỉnh mang lại sự linh hoạt và sáng tạo tối đa trong việc thiết kế.
    \item Quy trình thực thi các công việc mạnh mẽ cho phép các nhóm cộng tác hiệu quả trong mọi giai đoạn phát triển dự án.
    \item Hỗ trợ đa ngôn ngữ (hơn 50 ngôn ngữ) đảm bảo tính nhất quán giữa các khu vực địa lý bằng cách sử dụng giao diện bản địa hóa trực quan.
    \item Khả năng thiết lập, chạy và bắt đầu xây dựng trải nghiệm kỹ thuật số chỉ trong vài giây thông qua các dịch vụ đám mây.
\end{itemize}

\section{Docker}
Docker \cite{docker} là nền tảng phần mềm cho phép nhà phát triển tạo dựng, triển khai, và kiểm thử ứng dụng một cách nhanh chóng. Docker đóng gói phần mềm vào các đơn vị tiêu chuẩn hóa được gọi là container (vùng chứa), là nơi chứa đựng mọi thứ mà phần mềm cần để chạy, trong đó có thư viện, công cụ hệ thống, mã nguồn và thời gian chạy. Bằng cách sử dụng Docker, toàn bộ ứng dụng có thể được nhanh chóng triển khai và thay đổi vào bất kỳ môi trường nào, đồng thời cũng luôn biết chắc rằng mã nguồn sẽ chạy đúng như kỳ vọng. Việc ứng dụng Docker và Docker compose để đóng gói và xây dựng cả frontend cũng như backend giúp hệ thống hoạt động trơn tru, dễ đóng gói và vận chuyển hơn, giúp cho công đoạn phát triển phần mềm có thể diễn ra linh hoạt, không giới hạn thời gian và nơi làm.

\subsection{Cách thức hoạt động của Docker}
Docker hoạt động bằng cách cung cấp phương thức tiêu chuẩn để chạy các đoạn mã lệnh tạo ra phần mềm. Docker là hệ điều hành dành cho các container (đơn vị chứa). Cũng tương tự như cách máy ảo thực hiện ảo hóa phần cứng máy chủ (loại bỏ các công việc liên quan đến quản lý trực tiếp), có nghĩa là các đơn vị chứa sẽ ảo hóa hệ điều hành của máy chủ. Docker được cài đặt trên từng máy chủ và cung cấp các lệnh đơn giản mà nhà phát triển phần mềm có thể sử dụng để dựng, khởi động hoặc dừng các đơn vị chứa.

\subsection{Các tính năng nổi bật Docker}
Việc sử dụng Docker cho phép người dùng đóng gói và di chuyển mã nhanh hơn, tiêu chuẩn hóa hoạt động của ứng dụng, tiết kiệm chi phí bằng cách cải thiện khả năng tận dụng tài nguyên. Khi hệ thống chạy bằng Docker, người dùng sẽ nhận được một đối tượng duy nhất là sản phẩm từ việc thực thi mã lệnh dự án, có khả năng chạy ổn định ở bất kỳ đâu. Các cú pháp đơn giản để thực thi Docker sẽ cung cấp người quản lý toàn bộ quyền kiểm soát hệ thống. Việc ứng dụng nền tảng này đã tạo ra một hệ sinh thái bền vững để các công cụ và ứng dụng có thể sử dụng ngay. Ngoài ra, lập trình viên có thể sử dụng các đơn vị chứa làm cốt lõi để tạo dựng ứng dụng và các nền tảng khác. Docker khiến cho việc dựng và chạy các kiến trúc vi dịch vụ được phân phối, triển khai liên tục với quy trình đã được tiêu chuẩn hóa, xây dựng các hệ thống xử lý dữ liệu có quy mô cực kỳ linh hoạt cũng như tạo ra các nền tảng được quản lý đầy đủ. 

\subsection{Docker compose}
Docker compose \cite{dockercompose} là công cụ giúp định nghĩa và chạy các ứng dụng Docker có nhiều vùng chứa. Với Compose, nhà phát triển phần mềm sử dụng tệp YAML để cấu hình các dịch vụ của ứng dụng. Sau đó, chạy một lệnh duy nhất \texttt{docker compose up}, tất cả các dịch vụ đã cấu hình từ trước sẽ được tạo ra và bắt đầu đi vào hoạt động.

Compose có thể hoạt động trong mọi giai đoạn phát triển phần mềm: từ phát triển, triển khai, kiểm thử, đến cài đặt thành sản phẩm thực tế, cũng như trong giai đoạn của luồng tích hợp liên tục (\acrshort{ci} workflows). Nó cũng có các lệnh để quản lý toàn bộ vòng đời ứng dụng, bao gồm: chạy, dừng, và xây dựng lại dịch vụ. Bằng việc sử dụng Compose, lập trình viên có thể xem trạng thái của các dịch vụ, kiểm tra nhật ký hoạt động của các dịch vụ đang chạy, và chạy lệnh trực tiếp trong dịch vụ. Công cụ có các tính năng nổi bật như: các môi trường có thể chạy độc lập trên một máy chủ duy nhất, ứng dụng "volume", một cơ chế được Docker sử dụng để cung cấp khả năng lưu trữ liên tục, để bảo toàn dữ liệu khi tạo lại vùng chứa. Thêm vào đó, để giúp cho việc tối ưu, Compose sẽ chỉ tạo lại các vùng chứa khi có thay đổi, hỗ trợ các biến môi trường cho việc cấu hình Docker cũng như di chuyển các thành phần trong các vùng chứa giữa các môi trường phát triển phần mềm.

\section{Github}
Trong quá trình phát triển phần mềm theo dõi sắn KOICA, Github \cite{github} được sử dụng làm công cụ hỗ trợ cho việc quản lý mã nguồn của cả frontend và backend. Được Erlang do Tom Preston-Werner, Chris Wanstrath, và PJ Hyett lập trình bằng Ruby on Rails, Github chính thức ra mắt vào tháng 4 năm 2008 và tính đến 2023, công cụ đã đạt được 100 triệu người dùng, có tới hơn 330 triệu dự án được lưu trữ tại đây. Github là một hệ thống quản lý dự án và các phiên bản mã nguồn của dự án được nhiều tổ chức, doanh nghiệp hay cá nhân tin dùng. Các lập trình viên có thể tải về mã nguồn từ các dự án công khai trên Github. Tất cả mọi người đều có thể tạo tài khoản hoàn toàn miễn phí trên đó và tạo ra các kho chứa mã nguồn để có thể làm việc ở bất cứ đâu. Một trong các tính năng quan trọng nhất của Github là đơn giản hóa việc quản lý mã nguồn. Dự án khi được đẩy lên và lưu trữ trên Github, các thành viên có quyền truy cập đều có thể tải về, thay đổi và lưu trữ lại, toàn bộ các sửa đổi đều được công cụ ghi nhận và định danh, giúp nhóm phát triển có thể xác định được phần nào và do ai phát triển. Dự án cũng có thể chia thành các nhánh độc lập nhau và khi cần gộp chung lại, Github sẽ so sánh sự thay đổi giữa các bản và thực hiện việc tổng hợp lại thành một tập thống nhất. Thay vì lưu trữ toàn bộ từng phiên bản mã nguồn, nó chỉ lưu lại sự thay đổi nếu có và ghi chú liên quan mỗi lần người dùng đánh dấu bằng commit, làm giảm thiểu bộ nhớ lưu trữ, tối ưu luồng hoạt động, đây cũng là một điểm đặc biệt giúp Github nổi trội hơn so với các công cụ quản lý mã nguồn cũ. Nhờ vào công cụ này mà việc lưu trữ, vận chuyển mã nguồn trở nên linh động hơn cũng như tạo các phiên bản phần mềm có khả năng quay trở lại trong trường hợp phiên bản mới gặp lỗi cũng không còn là vấn đề. 

\section{Các framework, thư viện hỗ trợ frontend}

\subsection{Ant Design}
\acrfull{antd} \cite{antdesign} là một thư viện UI mã nguồn mở miễn phí do Alibaba của Trung Quốc tạo ra và hiện tại nó nhận được khoảng 85500 sao trên Github. Khi sử dụng \acrshort{antd} ta có thể dễ dàng dựng nhanh một trang đích (landing page) dựa vào các thành phần đã được xây dựng, cấu hình giao diện sẵn. Một trong các điểm nổi bật của \acrshort{antd} so với các thư viện UI khác là nó hỗ trợ đa dạng các cho các framework dùng cho frontend,.. trong khi Material UI lại chỉ hỗ trợ cho framework ReactJS hay Vuetify chỉ hỗ trợ cho VueJS thì \acrshort{antd} có thể hỗ trợ cho cả 2 và cho cả AngularJS. Vì là một thư viện mã nguồn mở miễn phí nên \acrshort{antd} có thể dễ dàng cài đặt theo tài liệu trên trang chủ. Dù hỗ trợ nhiều frontend framework khác nhau nhưng các thành phần của \acrshort{antd} cũng rất đa dạng và đầy đủ để nhà phát triển phần mềm có thể dễ dàng dựng lên các giao diện phức tạp. Một điểm mạnh khác của \acrshort{antd} là tài liệu trên trang chủ được viết chi tiết, rõ ràng và dễ hiểu, cùng với đó là rất nhiều ví dụ đi kèm cách sử dụng từng thuộc tính của mỗi thành phần, giúp đơn giản hóa việc tiếp cận, sử dụng và cấu hình lại các thành phần trong \acrshort{antd} cho người sử dụng. Lập trình viên khi muốn sửa lại các chủ đề hay cái thuộc tính gốc cốt lõi của \acrshort{antd} cũng không gặp khó khăn gì vì tất cả đều được hướng dẫn trong tài liệu đi kèm thư viện này, giúp linh hoạt hơn khi xây dựng các giao diện cho mọi thiết kế khó hay đặc thù. Sử dụng \acrshort{antd} trong dự án giúp cho việc xây dựng giao diện thân thiện với người dùng nhanh chóng và dễ dàng, đồng nhất trong thiết kế. Ngoài phần tử \acrshort{html} được thiết kế sẵn, dùng các phần tử riêng của \acrshort{antd} cũng góp phần phát triển dự án thuận tiện hơn, giảm chi phí về mặt thời gian, công sức viết mới mã lệnh.

\subsection{TailwindCSS}
TailwindCSS \cite{tailwindcss} là một utility-first \acrshort{css} framework (framework ưu tiên tiện ích dành cho \acrshort{css}), nó hỗ trợ phát triển xây dựng giao diện người dùng từ các bản thiết kế một cách nhanh chóng. Ngoài ra, cũng có điểm chung giống như Bootstrap là các tệp \acrshort{css} đã được cấu hình, viết mã sẵn cho các phần tử trên giao diện, nhưng điểm làm nó nổi bật hơn cả đó là lập trình viên có thể tùy biến phát triển các cấu hình sẵn có đó theo cách mà họ tự định nghĩa, sao cho phù hợp với mục đích sử dụng. Dưới đây là một số đặc điểm nổi bật của Tailwind:
\begin{itemize}
    \item \textbf{Miễn phí và dễ dàng sử dụng:} TailwindCSS là một framework \acrshort{css} mã nguồn mở miễn phí, có thể dễ dàng cài đặt và sử dụng theo tài liệu trên trang chủ. Tailwind có thể sử dụng với \acrshort{html} thuần hay các frontend framework nổi bật như ReactJS, VueJS, AngularJS,... Bởi vậy nên cộng đồng hỗ trợ người dùng lớn, giúp cho việc tìm kiếm câu trả lời khi gặp vấn đề trở nên dễ dàng và nhanh chóng hơn.
    \item \textbf{Loại bỏ rắc rối khi đặt tên lớp:} Một rắc rối khá lớn đối với bất cứ ai khi thiết kế lại thể hiện của các phần tử \acrshort{html} là việc đặt tên lớp. Trong một dự án lớn thì việc đó lại càng cần phải để ý hơn vì khi mà vô tình đặt trùng tên lớp thì có thể làm bố cục giao diện của trang web bị vỡ, hiện tượng này thường được gọi là "vỡ layout". Với việc sử dụng Tailwind, nhà phát triển có thể dễ dàng xử lý vấn đề này. Tailwind cung cấp hàng trăm lớp đơn giản, mỗi lớp thể hiện cho một thuộc tính CSS nên có thể dễ dàng sử dụng để thiết kế lại các phần tử mà không sợ bị ghi đè các thiết kế trước bằng \acrshort{css}. Đặc biệt tên các lớp của Tailwind đều đơn giản và theo quy luật, vậy nên chúng khá dễ để nhớ và sử dụng. 
    \item \textbf{Khả năng tùy biến cao:} Mặc dù có những cấu hình mặc định nhưng chúng ta có thể dễ dàng ghi đè những cấu hình mặc định đó trong tệp tailwind.config.js. Bằng việc này, ta có thể dễ dàng cấu hình lại các thuộc tính gốc để phù hợp với thiết kế như khoảng cách giữa các phần tử, phông chữ, cỡ chữ, chủ đề,... tránh việc lập trình cứng, không thể sửa đổi các thuộc tính \acrshort{css} như việc sử dụng đơn vị pixel cho đo khoảng cách, dẫn đến khó khăn trong tính linh hoạt của phần mềm trên các thiết bị hiển thị khác nhau, các việc liên quan đến bảo trì và tính khả mở.
    \item \textbf{Dễ dàng xây dựng các giao diện phức tạp:} Tailwind đơn giản hóa từng thuộc tính thành một lớp, từ đó người dùng có thể dễ dàng thiết kế cho từng phần tử, hiệu quả tương tự với việc sử dụng đặt CSS vào trong các phần tử \acrshort{html}, nhưng khi dùng Tailwind sẽ dễ dàng điều chỉnh lại thuộc tính khi cần, phù hợp cho việc dựng các giao diện khó. Tailwind cũng cung cấp sẵn các điểm dừng và lớp giả, giúp cho việc đáp ứng các thiết bị hiển thị trang web hay tạo ra hiệu ứng trở nên đơn giản hơn.
    \item \textbf{Tối ưu hóa bằng PurgeCSS:} Một lợi thế lớn của TailwindCSS là việc tối ưu hóa có thể được thực hiện bằng PurgeCSS, giúp giảm đáng kể kích thước tệp bằng cách quét các tệp \acrshort{html} và xóa các lớp không sử dụng, từ đó làm giảm kích thước dự án và xây dựng nhanh hơn.
\end{itemize}

Nhận thấy nếu như chỉ sử dụng \acrshort{antd} là không đủ để tạo ra sự khác biệt với các trang khác, nhờ việc ứng dụng thêm Tailwind, phần mềm có thêm những phần tử riêng đặc biệt. Khi kết hợp \acrshort{antd} và Tailwind với nhau, công đoạn viết mã lệnh cho frontend trở nên nhanh hơn nhưng vẫn đảm bảo tính thẩm mỹ cho giao diện và hiệu suất khi sử dụng.

\subsection{Axios}
Axios \cite{axios} là một \acrshort{http} client khá nhẹ, chỉ 11.2kb (Minified + Gzipped), được viết dựa trên Promises, và dùng để hỗ trợ cho việc xây dựng các ứng dụng gọi \acrshort{api} từ đơn giản đến phức tạp, có thể được sử dụng cả ở trình duyệt hay Node.js. Một trong các điểm đặc trưng của Axios so với fetch được tích hợp sẵn trong trình duyệt là khả năng cấu hình cho toàn bộ yêu cầu các tham số như phần mở đầu của yêu cầu, thời gian chờ đợi của yêu cầu, \acrshort{url} cơ sở để gọi \acrshort{api},... với việc này, các yêu cầu khi dùng sẽ gọn hơn và được xử lý tốt hơn. Một request sẽ được huỷ bỏ nếu quá thời gian chờ đã được lập trình viên thiết lập mà vẫn chưa nhận phản hồi. Với bộ chặn hữu ích của Axios, là phần trung gian có thể can thiệp vào yêu cầu của ứng dụng trước khi chúng được gửi đến cho máy chủ và là phần trung gian nhận kết quả trả về, có thể tác động tới kết quả đó trước khi chuyển tiếp cho ứng dụng xử lý, người dùng có toàn quyền kiểm soát, xử lý logic những việc liên quan đến tương tác máy chủ. Một ứng dụng thường thấy của bộ chặn Axios là việc đính kèm token vào yêu cầu trước khi chúng được gửi đi cho việc xác thực yêu cầu, và xử lý lỗi ở phản hồi trước khi trả ra kết quả hiển thị lên giao diện. Với việc tiền xử lý trước khi hiển thị, người dùng có thể dễ dàng xử lý chung các lỗi của phản hồi, giúp hạn chế việc trùng lặp mã lệnh, đồng bộ trong logic xử lý, đặc biệt nhà phát triển phần mềm có thể xử lý vấn đề token hết hạn mà không bắt người dùng phải đăng nhập lại, giúp tăng trải nghiệm người dùng. Với Axios, ta có thể dễ dàng huỷ bỏ một yêu cầu bằng cancelToken hay AbortController, điều này cũng giúp tiết kiệm tài nguyên, thời gian chờ, góp phần tăng hiệu năng của trang.

\subsection{Leaflet}
Leaflet \cite{leaflet} là một thư viện JavaScript mã nguồn mở, được sử dụng để xây dựng bản đồ có khả năng tương tác. Đây là một thư viện khá nhẹ, chỉ khoảng 42KB cho phần mã nguồn nhưng lại có đầy đủ tất cả các tính năng mà hầu hết các nhà phát triển cần. Leaflet chú trọng tới sự đơn giản trong các ứng dụng nhưng vẫn mang lại hiệu suất cao, và khả năng sử dụng linh hoạt trên các thiết bị hiển thị khác nhau. Leaflet được thiết kế chú trọng tới sự đơn giản, hiệu suất, và khả năng sử dụng, có thể mở rộng với nhiều tiện ích mở rộng. Nhờ công cụ này, bản đồ sắn tại tỉnh Tây Ninh có thể hiển thị trực quan và người dùng có thể tương tác dễ dàng với nó bằng việc chọn các điểm để xem thông tin khái quát dịch bệnh tương ứng. Việc cài đặt công cụ vào phần mềm cũng không khó khăn và dễ dàng trong việc ứng dụng nó.

\subsection{Quill}
Quill \cite{quill} là một trình soạn thảo \acrshort{wysiwyg} mã nguồn mở miễn phí được xây dựng cho các trang web. Với kiến trúc có thể mở rộng và các \acrshort{api} phản hồi nhanh chóng, lập trình viên hoàn toàn có thể tùy chỉnh nó để đáp ứng nhu cầu sử dụng. Một số tính năng được tích hợp sẵn bao gồm: nhập nội dung, chèn phương tiện truyền thông như ảnh, video, âm thanh, định dạng văn bản,... Bằng việc sử dụng Quill cho phần viết bài báo, người dùng phần mềm có thể hoàn toàn lựa chọn phong cách viết, chèn nội dung, hình ảnh, các phương tiện truyền thông như video, âm thanh tùy theo mong muốn mà không bị giới hạn, giúp cho bài báo trở nên phong phú, đa dạng hơn. Vì các tài liệu liên quan đến cài đặt và sử dụng công cụ đều được viết rõ ràng nên ứng dụng công nghệ vào dự án và triển khai cũng đơn giản, nhanh chóng.

\subsection{React router}
React router \cite{reactrouter} là một thư viện định tuyến tiêu chuẩn trong ứng dụng React, dùng để giữ cho giao diện của ứng dụng đồng bộ với \acrshort{url} trên trình duyệt, giúp định tuyến "luồng dữ liệu" trong ứng dụng một cách rõ ràng. Có nghĩa là, với \acrshort{url} này sẽ tương đương với một điều hướng cùng một giao diện được định nghĩa trong mã lệnh. Lợi ích lớn nhất của React router là hỗ trợ xây dựng \acrfull{spa}. Với công cụ này, việc điều hướng giữa các trang trở nên dễ dàng, nhanh chóng và đặc biệt là không cần phải tải lại trang, giúp cải thiện hiệu suất, trải nghiệm người dùng. Khi điều hướng, nó sẽ tìm đến thành phần tương ứng với \acrshort{url} đó và hiển thị chúng, những phần không thay đổi sẽ được giữ nguyên, điều này khiến cho trang web phản hồi nhanh chóng trước yêu cầu điều hướng từ người dùng và cũng giúp cho trang không phải tải lại toàn bộ trang để hiển thị lại. React router hỗ trợ rất nhiều các điều hướng với các điểm bắt sự kiện, nhờ đó việc xử lý điều hướng trở nên đơn giản hơn nhiều. Ngoài ra, cũng có thể dễ dàng xây dựng các điều hướng xác thực liên quan tới phân quyền người dùng trên máy khách. Một tính năng hữu dụng khác của React router là hỗ trợ lazy loading (tải lười), với tính năng này, trang sẽ chỉ tải những thành phần đang tiếp cận tới và hạn chế việc tải những gì chưa sử dụng, đối với các ứng dụng lớn thì việc này sẽ cải thiện hiệu năng trang đáng kể. Khi sử dụng công cụ này, trang được điều hướng đúng như mong đợi và việc tải lại trang được giảm thiểu tối đa, giúp làm giảm thời gian chờ đợi tải trang khi truy cập vào đường dẫn mới.
\end{document}