% !TeX root = ../main.tex
\documentclass[./../main.tex]{subfiles}

\begin{document}
\section{Mô tả vấn đề}

Cây sắn (hay còn gọi là khoai mì), từ lâu đã được người nông dân ở tỉnh Tây Ninh trồng và hàng năm đều chú trọng đến việc tăng diện tích canh tác, tăng năng suất. Loại cây trồng này đã giúp cho các hộ dân tăng khả năng kinh tế, tạo công ăn việc làm cho các lao động tại địa phương trong tỉnh. Theo thống kê, Tây Ninh là tỉnh có diện tích trồng sắn lớn thứ hai cả nước sau tỉnh Gia Lai, nhưng xét về năng suất, chất lượng, và số lượng nhà máy chế biến tinh bột thì tỉnh được xếp vào vị trí lớn nhất nước \cite{tayninh1}. Qua dự báo về thị trường xuất khẩu, Giám đốc Sở Nông nghiệp và Phát triển nông thôn tỉnh Tây Ninh cho rằng, nhu cầu thế giới về sắn và các chế phẩm từ sắn vẫn sẽ tăng cao, do các nước phát triển chuyển sang sử dụng năng lượng sạch, và trong năng lượng sạch đó có năng lượng từ Ethanol, một hợp chất hữu cơ mà sắn có thể tạo ra. Năm 2022, tuy dịch bệnh khảm lá sắn diễn ra nghiêm trọng tại tỉnh này, nhưng nhờ vào kinh nghiệm trong canh tác cũng như việc ứng dụng các tiến bộ khoa học kỹ thuật vào quy trình sản xuất, năng suất sắn ở nơi đây vẫn đạt trên 30 tấn/ha \cite{tayninh2}. Như vậy, có thể thấy dù là trước đây, hiện tại, hay tương lai, thì ngành nông nghiệp sắn ở tỉnh Tây Ninh vẫn là ngành đặc biệt quan trọng, được chú trọng và đầu tư đến.

Ngành nông nghiệp sắn không chỉ cần quan tâm đến giống cây trồng, thời vụ trồng, mà sắn là loại cây trồng lấy dưỡng chất chủ yếu từ đất nên việc bón phân, bổ sung dưỡng chất cho đất là yêu cầu bắt buộc nếu năng suất tốt. Ngoài ra, sự biến đổi khí hậu đang diễn ra, các giống bệnh lạ xuất hiện trên cây, cách phòng ngừa và trị bệnh, hay kỹ thuật trồng, tin tức xoay quanh loại cây này cũng là những điều phải chú ý tới. Sau khi sắn đến mùa thu hoạch, các chế phẩm từ sắn đã được hoàn thiện thì nhu cầu về thương mại như mua bán, trao đổi cũng được người nông dân quan tâm đến.

Nhận thấy việc nắm bắt thông tin kịp thời về ngành nông nghiệp này, có thể tìm kiếm dữ liệu về sắn hay liên quan tới sắn như bệnh, các vùng xảy ra dịch bệnh, nguồn cung cấp giống sắn, hay đề xuất trao đổi mua, bán sắn có thể giúp ích cho người nông dân, phần mềm quản lý sắn KOICA đã ra đời. Bằng cách tổng hợp lại những thông tin trên và cập nhật các tin tức mới nhất, phần mềm không chỉ giúp người nông dân có thể tăng năng suất vụ mùa sắn, mà còn tạo cơ hội cho các giao dịch liên quan tới loại cây trồng này, có tiềm năng mở rộng quy mô trồng trọt, kinh doanh, giúp phát triển kinh tế cho cá nhân và kinh tế khu vực. Khóa luận này sẽ đi sâu tìm hiểu, phân tích, thiết kế, và xây dựng phần mềm quản lý sắn KOICA, phần mềm có tính khả mở và cần được bảo trì thường xuyên để đảm bảo khả năng hoạt động trong trạng thái tốt nhất.

\section{So sánh với phần mềm Seedtracker}
Phần mềm Seedtracker \cite{seedtracker} được tổ chức phi lợi nhuận \acrshort{iita} phát triển riêng cho một số nước Châu Phi ví dụ như Tanzania hay Nigeria. Đó là những nơi mà sắn được coi là một trong những cây nông sản được quan tâm và chú trọng đầu tư, có tầm quan trọng lớn trong số các sản phẩm nông sản ở những quốc gia này. Phần mềm có nhiều tính năng, cơ sở dữ liệu đầy đủ với nhiều trường thông tin liên quan đến danh sách người cung cấp giống sắn, danh mục các giống sắn phổ biến (+25 loại giống sắn với nhiều thông tin chi tiết), các công cụ hỗ trợ thu hoạch và chức năng hỗ trợ tổng hợp thông tin quản lý nhà nước về cây sắn. Phần mềm seedtracker có một số tính năng khác như cho phép người mua, bán, cung cấp giống sắn điền các thông tin cá nhân vào biểu mẫu và có thể chia sẻ cho người khác khi sử dụng phần mềm. Các trường thông tin về người cung cấp giống sắn gồm địa chỉ, giống sắn, lượng cung,... cho phép người dùng nắm được thông tin liên hệ và mua giống sắn phù hợp. Các cơ sở dữ liệu, ví dụ như cơ sở dữ liệu về giống sắn được thiết kế để người sử dụng có thể đọc được các thông tin về mỗi giống sắn phổ biến, với các trường thông tin như: đặc tính, sản lượng, các thông số nông học, hình ảnh các giai đoạn phát triển, sự phù hợp của giống sắn với địa phương,...

Phần mềm sắn KOICA sẽ mô phỏng một số chức năng cơ bản giống Phần mềm Seedtracker nhưng cũng sẽ có một số bổ sung theo đặc thù Việt Nam, và đặc biệt sử dụng AI để cung cấp cho người sử dụng tính năng chẩn đoán bệnh. Các thông tin được nhóm nghiên cứu \acrshort{vkist} kết hợp với Viện nghiên cứu trí tuệ nhân tạo của trường Đại học Công nghệ - Đại học Quốc gia Hà Nội thu hoạch trong chuyến công tác tới tỉnh Tây Ninh đầu tháng 10 năm 2022 trong cơ sở dữ liệu của phần mềm theo dõi sắn KOICA sẽ dùng làm cơ sở dữ liệu cho hệ thống, song song với quá trình phát triển phần mềm và cả trong giai đoạn hoàn thiện các tính năng của phần mềm.

\section{Mục tiêu đặt ra}
Mục tiêu đặt ra cho sản phẩm giải quyết bài toán trên là một phần mềm quản lý có tối thiểu các chức năng xem, thêm, sửa, xóa dữ liệu, có tính bảo mật thông tin cho người dùng, bảo đảm tính toàn vẹn dữ liệu, có khả năng chẩn đoán bệnh bằng hình ảnh. Về giao diện, phần mềm cần có sự đồng nhất trong thiết kế, bố cục trang được sắp xếp hợp lý, giúp người dùng có thể dễ dàng sử dụng. Về mặt thông tin, dữ liệu trên hệ thống thì cần được hiển thị một cách trực quan, đầy đủ khi truy cập vào trang xem chi tiết, để người sử dụng nắm bắt thông tin kịp thời, những dữ liệu không công khai cần được bảo mật và chỉ cho phép những người có quyền được tác động tới. Trong những công việc cần thay đổi cơ sở dữ liệu, phần mềm cũng phải được duy trì, không bị thời gian chết và không làm ảnh hưởng tới các thao tác khác của hệ thống. Với sản phẩm đạt đủ các tiêu chí trên, người dùng có thể thuận tiện sử dụng phần mềm cho mục đích cá nhân hay hướng xa hơn là phát triển cộng đồng sắn.

Khóa luận này sẽ làm rõ các quá trình từ khâu phân tích, thiết kế hệ thống, đến quá trình thực thi triển khai phần mềm quản lý sắn KOICA, kiểm thử và ứng dụng sản phẩm. Từ đó đưa ra các nhận xét về kết quả đã đạt được, những hạn chế vẫn còn tồn tại, và cuối cùng là định hướng phát triển cho phần mềm.

\section{Đối tượng và phạm vi đề tài}
\subsection{Đối tượng}
Hệ thống sẽ đi vào nghiên cứu về các giống sắn và những thông tin liên quan đến loại cây trồng này, dữ liệu về dịch bệnh sắn sẽ được địa phương hóa để tập trung vào tỉnh Tây Ninh. Dữ liệu về sắn, bệnh trên cây sắn được tỉnh Tây Ninh cung cấp tương đối đầy đủ và có thể bổ sung thêm. Những thông tin liên quan tới nguồn cung sắn, việc mua, bán sắn trên màn thương mại sắn được người dùng cập nhật theo nhu cầu nếu có. Những bài báo về sắn và các vấn đề xung quanh được quản trị viên thường xuyên cập nhật, và có thể thêm mới bởi người dùng.

Bài toán được đặt ra bởi Viện Khoa học \& Công nghệ Việt Nam - Hàn Quốc, đối tượng người sử dụng hướng đến là những người có mong muốn tìm hiểu thêm về sắn hay các vấn đề liên quan, có nhu cầu giao dịch kinh tế mặt hàng này, những người nông dân trồng sắn, và các cán bộ cần quản lý, tổng hợp thông tin liên quan tới loài cây này ở tỉnh Tây Ninh.

\subsection{Phạm vi đề tài}
Phạm vi nghiên cứu của đề tài này sẽ tập trung vào ngành nông nghiệp trồng sắn ở tỉnh Tây Ninh. Thời gian nghiên cứu và phát triển sản phẩm từ tháng 10 năm 2022 và liên tục được cập nhật tính tới tháng 4 năm 2023. Trong đó, dành khoảng hai tháng để phân tích yêu cầu và lựa chọn các công nghệ sử dụng. Phạm vi của khóa luận tốt nghiệp là toàn bộ quá trình đi từ
bài toán đến các bước phát triển và hoàn thiện hệ thống để có thể ứng dụng vào thực tiễn. 

\section{Cấu trúc khóa luận}
Khóa luận này có 6 chương, bao gồm:
\begin{itemize}
    \item \textbf{Chương 1: Mở đầu}\\ Tổng quan về khóa luận tốt nghiệp sẽ được trình bày trong chương này. Nội dung trong chương sẽ bao gồm: giới thiệu bài toán, so sánh với phần mềm tương tự, mục tiêu, và phạm vi đề tài. Toàn bộ các chương sau đều nhằm làm rõ hơn các nội dung được nêu trên.
    \item \textbf{Chương 2: Cơ sở lý thuyết}\\ Chương này sẽ trình bày về cơ sở lý thuyết, và các công nghệ sử dụng để xây dựng phần mềm. Thêm vào đó, còn nêu ra việc ứng dụng các công nghệ kỹ thuật vào bài toán như thế nào, lợi ích của việc sử dụng chúng trong việc giải quyết vấn đề.
    \item \textbf{Chương 3: Phân tích yêu cầu}\\ Chương trình bày tổng thể và làm rõ các yêu cầu chức năng và các yêu cầu phi chức năng về hệ thống giúp giải quyết bài toán đã được đề ra trong chương mở đầu. Từ đó, đi sâu phân tích, mô tả chi tiết các ca sử dụng cũng như các luồng hoạt động, dựa vào các phân tích để tạo ra biểu đồ tuần tự tương ứng.
    \item \textbf{Chương 4: Thiết kế hệ thống}\\ Sau khi có được các ca sử dụng và luồng hoạt động chính, chương 4 sẽ đi sâu vào miêu tả, làm rõ các khía cạnh trong việc thiết kế, bao gồm thiết kế hệ thống, thiết kế backend, thiết kế frontend.
    \item \textbf{Chương 5: Triển khai, cài đặt và kiểm thử}\\ Từ các bản thiết kế và dựa vào yêu cầu bài toán, chương này sẽ nêu ra cách triển khai và cài đặt phần mềm, rồi dựa trên các tiêu chí đánh giá để kiểm thử hệ thống đã đạt yêu cầu chưa, những điểm cần cải thiện.
    \item \textbf{Chương 6: Kết luận}\\ Để tổng kết lại khóa luận tốt nghiệp, chương này sẽ tóm tắt lại các bước phát triển phần mềm, những điểm mạnh và hạn chế, những đóng góp thực tiễn mà phần mềm mang lại.
\end{itemize}

\end{document}